\section{Introduction} \label{sec:intro}
Latent viral reservoirs within HIV-1 infected individuals are a major obstacle on the path to cure for HIV \citep{Pace11}. Reservoirs are cells that contain integrated viral DNA but have entered a dormant state in which they not only have a lowered rate of virion production, but can also persist within a patient for several years. Even though highly active antiretroviral therapy (HAART) can reduce a patient's viral load to below detectable levels, if a patient stops treatment latent reservoirs may reseed an infection \citep{Joos08, Pomerantz03, Richman09}.

Due to selective pressures in the host's immune system and HIV's short generation time, HIV evolves very quickly throughout the course of infection \citep{Alizon13, Shankarappa99, Rambaut04}. Once a virus has integrated itself into a cell, the rate at which it accumulates mutations becomes negligible compared to the rate at which the active population in plasma evolves.

Phylogenetically, within a chronogram latency manifests as a difference between a sequence's sampling time and it's predicted sampling time (figure \ref{fig:latenttree}). For sequences that are from the active population virus, this assumption is reasonable, however for sequences suspected to be latent, it is not true that the sampling time represents the time at which evolution stops.

Not much is known about the timing of the establishment of latent reservoirs during the course of infection. The chronology of the establishment of reservoirs is of interest because the distribution of ages of the stored virus(es) may provide information as to the types of adaptations that the virus may have accumulated. Specifically, it may have an influence on how those infected cells react to immune-mediate or therapeutic treatments. 

There have been studies that mathematically model latency with ODES. There have been some studies aim to characterize potentially latent samples. There has been little work that evaluates the time of establishment of sequences. However, there hasn't been much work assigning dates to sequences that are suspected to be latent (there has been work on detecting latent reservoirs \cite{Immonen14}). /anote{clean this up}

We propose a simple framework that utilizes the assumption of a strict molecular clock on the evolution of HIV within host -- out method extracts timing information from the phylogenetic relationship between the virus sampled at various time-points along the infection. We first construct a phylogeny for plasma and PBMC samples from a patient then calibrate a strict clock to that tree based  solely on the dates of the plasma sequences. The root of this phylogeny is typically unknown, so we employ either root-to-tip regression, or outgroup rooting to find a candidate root. We assume that PBMC cells are latent, as they contain only integrated cells, and therefore have stopped evolving. We don't try to do any classification of these cells. To assess the sensitivity of our method, we first simulated sequence data with no latent behavior and evaluated the accuracy of the reconstruction. We then simulated latent behavior under a simplified model of latency \citep{Immonen14}. To mimic the types of results we expected from our first test, we used a previously published set of data to evaluate the reconstruction of dates from only plasma sequences \citep{McCloskey14}. Finally, we tested our methodology on a data-set comprised of patients that had longitudinal samples of both PBMC and plasma sequences.  

\begin{figure} \label{fig:latenttree}
	\centering
	\scalebox{5}{%LaTeX with PSTricks extensions
%%Creator: inkscape 0.91
%%Please note this file requires PSTricks extensions
\psset{xunit=.5pt,yunit=.5pt,runit=.5pt}
\begin{pspicture}(85,53)
{
\newrgbcolor{curcolor}{0 0 0}
\pscustom[linewidth=0.38699999,linecolor=curcolor]
{
\newpath
\moveto(77.921134,4.13898)
\lineto(5.1912118,4.13898)
\lineto(5.1912118,20.207747)
}
}
{
\newrgbcolor{curcolor}{0 0 0}
\pscustom[linewidth=0.38699999,linecolor=curcolor]
{
\newpath
\moveto(65.799478,25.564002)
\lineto(29.434525,25.564002)
\lineto(29.434525,36.276514)
}
}
{
\newrgbcolor{curcolor}{0 0 0}
\pscustom[linewidth=0.38699999,linecolor=curcolor]
{
\newpath
\moveto(41.556173,46.989025)
\lineto(29.434525,46.989025)
\lineto(29.434525,36.276514)
}
}
{
\newrgbcolor{curcolor}{0 0 0}
\pscustom[linewidth=0.38699999,linecolor=curcolor]
{
\newpath
\moveto(29.434525,36.276514)
\lineto(5.1912118,36.276514)
\lineto(5.1912118,20.207747)
}
}
{
\newrgbcolor{curcolor}{0 0 0}
\pscustom[linewidth=0.38699999,linecolor=curcolor]
{
\newpath
\moveto(4.4639148,20.207747)
\lineto(5.1912118,20.207747)
}
}
{
\newrgbcolor{curcolor}{0 0 0}
\pscustom[linestyle=none,fillstyle=solid,fillcolor=curcolor]
{
\newpath
\moveto(81.87934379,5.47198716)
\lineto(81.87934379,5.02049697)
\curveto(81.73520608,5.15474288)(81.58117656,5.25507403)(81.41725524,5.32149043)
\curveto(81.25474704,5.38790683)(81.08164047,5.42111503)(80.89793554,5.42111503)
\curveto(80.53617814,5.42111503)(80.25920764,5.31018551)(80.06702402,5.08832648)
\curveto(79.8748404,4.86788057)(79.77874859,4.54851662)(79.77874859,4.13023462)
\curveto(79.77874859,3.71336575)(79.8748404,3.3940018)(80.06702402,3.17214277)
\curveto(80.25920764,2.95169685)(80.53617814,2.84147389)(80.89793554,2.84147389)
\curveto(81.08164047,2.84147389)(81.25474704,2.87468209)(81.41725524,2.94109849)
\curveto(81.58117656,3.00751489)(81.73520608,3.10784604)(81.87934379,3.24209195)
\lineto(81.87934379,2.79484111)
\curveto(81.72955362,2.69309684)(81.5705782,2.61678864)(81.40241754,2.5659165)
\curveto(81.23566999,2.51504437)(81.05903063,2.4896083)(80.87249947,2.4896083)
\curveto(80.39345355,2.4896083)(80.01615188,2.63586569)(79.74059449,2.92838046)
\curveto(79.4650371,3.22230834)(79.3272584,3.6229264)(79.3272584,4.13023462)
\curveto(79.3272584,4.63895596)(79.4650371,5.03957402)(79.74059449,5.33208879)
\curveto(80.01615188,5.62601668)(80.39345355,5.77298062)(80.87249947,5.77298062)
\curveto(81.06185686,5.77298062)(81.23990933,5.74754455)(81.40665688,5.69667242)
\curveto(81.57481755,5.6472134)(81.73237985,5.57231831)(81.87934379,5.47198716)
\closepath
}
}
{
\newrgbcolor{curcolor}{0 0 0}
\pscustom[linestyle=none,fillstyle=solid,fillcolor=curcolor]
{
\newpath
\moveto(67.94040944,25.49054575)
\lineto(67.94040944,24.41103725)
\lineto(68.57982581,24.41103725)
\curveto(68.79428027,24.41103725)(68.95281869,24.45511225)(69.05544107,24.54326224)
\curveto(69.15937912,24.63272791)(69.21134815,24.76889991)(69.21134815,24.95177825)
\curveto(69.21134815,25.13597227)(69.15937912,25.27148644)(69.05544107,25.35832076)
\curveto(68.95281869,25.44647075)(68.79428027,25.49054575)(68.57982581,25.49054575)
\lineto(67.94040944,25.49054575)
\closepath
\moveto(67.94040944,26.70227923)
\lineto(67.94040944,25.81420095)
\lineto(68.53048813,25.81420095)
\curveto(68.72520752,25.81420095)(68.86993138,25.85038191)(68.96465973,25.92274385)
\curveto(69.06070376,25.99642145)(69.10872577,26.10825353)(69.10872577,26.25824009)
\curveto(69.10872577,26.40691097)(69.06070376,26.51808522)(68.96465973,26.59176282)
\curveto(68.86993138,26.66544043)(68.72520752,26.70227923)(68.53048813,26.70227923)
\lineto(67.94040944,26.70227923)
\closepath
\moveto(67.54176097,27.02988145)
\lineto(68.56009074,27.02988145)
\curveto(68.86401086,27.02988145)(69.0982004,26.96672921)(69.26265934,26.84042474)
\curveto(69.42711828,26.71412028)(69.50934775,26.53453111)(69.50934775,26.30165725)
\curveto(69.50934775,26.12141025)(69.46724626,25.97800205)(69.38304328,25.87143266)
\curveto(69.29884031,25.76486326)(69.17516718,25.69842185)(69.01202391,25.67210842)
\curveto(69.20805897,25.63000693)(69.36001903,25.54185694)(69.4679041,25.40765844)
\curveto(69.57710484,25.27477562)(69.6317052,25.10834317)(69.6317052,24.90836109)
\curveto(69.6317052,24.64522679)(69.54223954,24.44195553)(69.36330821,24.29854734)
\curveto(69.18437688,24.15513914)(68.92979444,24.08343504)(68.59956088,24.08343504)
\lineto(67.54176097,24.08343504)
\lineto(67.54176097,27.02988145)
\closepath
}
}
{
\newrgbcolor{curcolor}{0 0 0}
\pscustom[linestyle=none,fillstyle=solid,fillcolor=curcolor]
{
\newpath
\moveto(68.8325765,47.98448198)
\lineto(68.33071752,46.62360158)
\lineto(69.33626709,46.62360158)
\lineto(68.8325765,47.98448198)
\closepath
\moveto(68.62377386,48.34897081)
\lineto(69.04321075,48.34897081)
\lineto(70.08539237,45.6143888)
\lineto(69.70075592,45.6143888)
\lineto(69.45165803,46.31589242)
\lineto(68.21898978,46.31589242)
\lineto(67.96989189,45.6143888)
\lineto(67.57976063,45.6143888)
\lineto(68.62377386,48.34897081)
\closepath
}
}
{
\newrgbcolor{curcolor}{0 0 0}
\pscustom[linewidth=0.39463133,linecolor=curcolor,linestyle=dashed,dash=0.7892628 0.7892628]
{
\newpath
\moveto(41.217255,46.97727)
\lineto(65.885506,46.9308)
}
}
\rput(42,52){\psscalebox{0.3}{$t$}}
\rput(66,52){\psscalebox{0.3}{$t^\prime$}}
\end{pspicture}
}
	\caption[Example of latent behavior]{The dotted line in the above figure is an example of latency. Sequence A was archived at time $t$, yet was collected at the same time as sequence B, at time $t^\prime$, a drift of $t^\prime - t$ is expected.}
\end{figure}
