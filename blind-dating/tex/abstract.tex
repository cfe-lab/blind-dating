Background: The ability of HIV to persist within latent cellular reservoirs represents a major barrier to cure. The timing of establishment of individual viral reservoirs over the infection course could influence their susceptibility to elimination by immune-mediated or therapeutic approaches. However, “dating” methods to accurately estimate the age of reservoir sequences remain scarce. We propose a simple method to date suspected reservoir sequences using phylogenetic approaches. 

Method: Simulated sequence data for model validation were generated using INDELible version 1.03.  Published longitudinal clonal sequences from untreated HIV-infected individuals with estimated dates of infection were obtained from the Los Alamos National Laboratory database. Maximum-likelihood phylogenies were reconstructed with PhyML. Phylogenies were rooted by determining the location of the root that minimized the root-mean-square error between root-to-tip distances and known dates of sampling. The root-to-tip distances of latent sequences were mapped to the optimal regression line to estimate their establishment date, which was assumed to precede their sample dates by an unknown amount.
 
Results: We validated the root-to-tip method using simulated data and published longitudinal clonal sequence datasets from untreated HIV-1 infected individuals with known infection dates. For each dataset, arbitrary selections of up to 10% of sequences were handled as latent by censoring their respective sample dates.  The method accurately recovered these missing sample dates when the phylogeny conformed to a strict molecular clock, as was the case for simulated data (SOME KIND OF PERFORMANCE METRIC HERE?  MEAN RMSE?). A strict clock model could not be rejected in the majority of empirical within-host data sets. When applied to HIV DNA sequences in phylogenies containing dated HIV RNA sequences, we observed that the predicted dates tended to precede dates of sampling, which was consistent with latency (MAYBE SOME METRIC HERE LIKE MEAN DISCORDANCE).

Conclusions:
Given a known phylogeny comprised of longitudinal plasma HIV-1 RNA sequences, the establishment dates of “unknown” (reservoir) sequences can be reliably estimated when within-host HIV sequence evolution conforms to a molecular clock. Future studies will test this method on empirically derived longitudinal within-host HIV deep sequence data and extend the method to work under alternative clock models.

Keywords: 
HIV, Latency, Cellular Reservoirs, Phylogenetics, Linear Regression
