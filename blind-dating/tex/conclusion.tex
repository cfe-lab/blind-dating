\section * {Conclusion} \label{sec:conclusion}
In this work we showed that a phylogeny's clock, when reliable, is a valuable source of information with respect to dating sequences that are suspected to show latent behaviour in patients infected with HIV-1. 
Via simulation, we built phylogenies that adhered to clock-like behaviour, with and without latent sequences.
From those, were able to reliably reconstruct dates for both active and latent sequences, rooting the trees via root-to-tip regression.
Then, using real data, we performed analogous reconstructions, using both root-to-tip regression and outgroup rooting, and showed similar results.

However, there are ways in which this method could be improved. 
These results hinge on the strong assumption of a molecular clock over the data, which is not always true. 
It may be possible to adapt a maximum likelihood relaxed clock methodology with this approach, giving more flexibility over noisy or unreliable data. 
Additionally, we would like to apply the methodology to next gen sequence data -- we would like to see if mass amounts of short read data contain sufficient information to calibrate the clock.
